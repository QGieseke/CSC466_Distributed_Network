\documentclass{article}

\oddsidemargin  -0.25in
\evensidemargin -0.25in
\marginparwidth 40pt
\marginparsep 10pt
\topmargin -0.5in
\headsep 0pt
\textheight 9.5in
\textwidth 7.0in

\begin{document}

\title{Dumb Distributed Network}

\author{Quinn Gieseke (V00884671), CSC 466}

\date{\today}

\maketitle
\newpage
%Per smart node sensor reads per second:	~4200 w/ replication across nodes (Fully connected)
%(~13 or more)
%Description (10 min)

\section{Final motivation}
\begin{itemize}
	\item Monitor status of sensors
	\item To allow for every smart node to have as current a value for every sensor in the network as possible.
	\item Ensure that data is transmitted reliably between smart nodes.
	\item Ensure that data becomes stale quickly.
	\item Must ensure that stale data does not persist, ie gets distributed between nodes.
	\item Be scalable.
	\item Have low overhead.
\end{itemize}

\newpage

\section{Sensor Description}
\begin{itemize}
	\item Produces single integer value for data (can be expanded later).
	\item Monotonic counter per data output.
\end{itemize}

\subsection{I/O}
	Write only datagram.

\newpage

\section{Smart Node Description}
\begin{itemize}
	\item Monotonic counter per sensor read.
	\item Keeps state for every sensor 
		\subitem Last data value 
		\subitem Last sensor tick 
		\subitem Approximate last node tick
	\item Bitmap for stale sensors
\end{itemize}
\subsection{I/O}
	Dedicated threads for sensor reads and node reads.

	Produces a heartbeat pulse every second which includes(for every up to date sensor):
\begin{itemize}
	\item Sensor data
	\item Sensor tick for data
	\item Negative offset for node tick
\end{itemize}
	
\newpage

\section{Simulator Description}
\begin{itemize}
	\item Routes all information through named pipes
	\item Maps locality, so that sensors and nodes only communicate through spacially near nodes. %updates randomly once per second
	\item Incurs message drop at set rate.
\end{itemize}

\newpage

\section{Tuneable parameters}
\begin{itemize}
	\item Number of nodes and sensors %(Quickly exhausts a single computers resources
	\item How long it takes for a bit of sensor data to become stale (in Node Ticks) %(Ideally set to around 2-3x the heartbeat timeout
	\item Maximum size of spacial map
	\item	Maximum distance for sensors/nodes to be able to transmit	(Ideally recieve roughly full coverage between the two)
	\item	Drop rate for communication.
\end{itemize}

\newpage
\section{Results}
	Each smart node is capable of sustaining ~4200 sensor reads per second (scaled to 4 smart nodes).
	
	To get around this, each sensor created new results at 1000 times per second, instead of upping the number of sensors to reduce operating system overhead.
	
	All further tests were restricted by my CPU or simulator code, not the actual protocol.


\newpage
\section{Further steps}

%	Easy:
%		Increasing the amount of data to share beyond a single int, would just require the sensor/smart node agreeing on data trasfer format
%		Testing on distributed hardware for more realistic results.
%		Check for tick overflows for long-running simulations
%	Medium:
%		More accurate performance data - will drastically increase overhead of processes
%	Hard:
%		Allow for nodes to share stale data in order to have a more whole view of past time steps (closer to full database replication)
%		Allow for data outflow from one or more nodes for long term storage


\newpage
\section{URL}
https://github.com/QGieseke/CSC466\_Distributed\_Network

\end{document}
